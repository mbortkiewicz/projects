% !TEX encoding = UTF-8 Unicode
\documentclass{article}

\usepackage{polski}
\usepackage[utf8]{inputenc}
\usepackage{subfig}
\usepackage{multirow}
\usepackage{graphicx}

\usepackage[a4paper, left=2.5cm, right=2.5cm, top=3.5cm, bottom=3.5cm, headsep=1.2cm]{geometry}

\linespread{1.3}
\begin{document}
	
	\begin{titlepage}
		\centering
		{\scshape\LARGE Politechnika Wrocławska \par}
		{\scshape\Large Katedra Informatyki Technicznej\par}
		
		\vspace{1cm}
		{\scshape\Large Inżynieria Oprogramowania\par}
		\vspace{1.5cm}
		{\huge\bfseries \par}
		\vspace{2cm}
		{\Large\itshape Magdalena Biernat\par}
		{\Large\itshape Mateusz Bortkiewicz\par}
		\vfill
		Opiekun\par
		prof. dr hab. inż. Jan Magott 
		
		\vfill
		{\large \today\par}
	\end{titlepage}
	\newpage
	\subsection{zwrot()}
	Diagram sekwencji zwrotu wywołuje funkcją zwrot(klient : TKlient, produkty : TProdukty[]) wywołąnie obiektu TWypozyczalnia.
	
	\subsection{oblicz\_koszt\_wypozyczenia()}
	Diagram sekwencji PU Obliczanie terminu zwrotu i kosztu wypożyczenia wywołuje funkcję oblicz\_koszt\_wypozyczenia(produkty : TProdukt[], liczba dni : int).
	
	\subsection{przyjmij\_towar()}
	Diagram sekwencji PU Przyjęcia towaru wywołuje funkcję przyjmij\_towar(produkty : TProdukt[], wypozyczenie : TWypozyczenie)
	
	\subsection{szukaj\_TWypozyczenie()}
	Diagram sekwencji Szukanie wypozyczenia wywołuje funkcję szukaj\_TWypozyczenie(produkt : TProdukt, klient : TKlient)
	
	\subsection{wypozyczenie()}
	Diagram sekwencji wypozyczenie wywołuje funkcję wypozyczenie(klient : TKlient, produkty : TProdukt[], ilosc\_dni : int)
\end{document}